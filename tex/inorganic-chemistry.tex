\chapter{Preparation and Properties of Metal Compounds}

The ordinary level syllabus includes the study and preparation of many simple inorganic compounds. This chapter discusses the preparation of metal carbonates, oxides, hydroxides, sulphates, and sulphides. These preparations help students understand the connections between different chemicals and to observe the variety of reactions in inorganic chemistry.

\section{Preparation of Metal Carbonates}

Insoluble metal carbonates may be easily formed by the reaction of a sodium carbonate solution with a solution of the desired metal cation. Two such examples provided in one activity -- the precipitation of magnesium and copper carbonate. Both soluble and insoluble metal carbonates may be formed by the reaction of a soluble metal hydroxide with carbon dioxide gas. The activity in this section offers an example of this process -- the production of calcium carbonate.

All of these preparations may be performed by students individually or working in small groups. The chemicals involved are relatively safe compared with other activities in this chapter.

\subsection{Preparation of Metal Carbonates by precipitation.}

\subsubsection*{Learning Objectives}
\begin{itemize}
\item{To prepare metal carbonate by precipitation reaction.}
\end{itemize}

\subsubsection*{Materials}
Epsom salt (magnesium sulphate)* and/or copper (II) sulphate*, washing soda (sodium carbonate)*, funnel*, cotton wool, beakers*, spatulas

\subsubsection*{Preparation}
\begin{enumerate}
\item{Stuff cotton wool into the funnel to plug the hole at the bottom.}
\end{enumerate}

\subsubsection*{Activity Procedure}
\begin{enumerate}
\item{In one beaker, add 2 spoons of magnesium sulphate or 1 spoon of copper sulphate to about 100 mL of water. Stir with a spatula until completely dissolved.}
\item{In a second beaker, add 2 spoons of sodium carbonate to about 100 mL of water. Stir with a spatula until completely dissolved.}
\item{Add the sodium carbonate solution to the magnesium sulphate / copper sulphate solution. A precipitate should form immediately.}
\item{Place the funnel on top of another beaker and pour the solution with the precipitate into the funnel. A clean liquid (filtrate) should collect in the beaker below the funnel. A solid precipitate should collect in the funnel.}
\item{Leave the funnel to sit until all of the liquid has moved through. Then remove the solid accumulated in the funnel and leave to dry.}
\end{enumerate}

\subsubsection*{Results and Conclusion}
When magnesium sulphate solution is mixed with sodium carbonate solution, magnesium carbonate precipitates. When copper sulphate solution is mixed with sodium carbonate solution, copper carbonate precipitates. This demonstrates the preparation of metal carbonates by precipitation reactions.

\subsubsection*{Clean Up Procedure}
\begin{enumerate}
\item{Save the dried precipitate (magnesium carbonate or copper carbonate) for a future experiment.}
\end{enumerate}

\subsubsection*{Discussion Questions}
\begin{enumerate}
\item{What did you observe when the solutions were mixed?}
\item{What is chemical precipitated?}
\item{Write the ionic equation for the formation of the product.}
\end{enumerate}

\subsubsection*{Notes}
This experiment deals with the solubility of different compounds. Magnesium sulphate and copper sulphate are both soluble in water, as is sodium carbonate. Magnesium carbonate and copper carbonate are both insoluble. As soon as magnesium and carbonate ions meet in solution, they form a white precipitate. As soon as copper and carbonate ions meet in solution, they form a blue precipitate. Many other metal carbonates may be prepared with this method.

\subsection{Preparation of Metal Carbonate by Reaction of Carbon Dioxide with Alkali}

\subsubsection*{Learning Objectives}
\begin{itemize}
\item{To understand how carbon dioxide reacts with an alkali solution.}
\end{itemize}

\subsubsection*{Background Information}
Carbon dioxide reacts with alkalis to form metal carbonates. Some carbonates are soluble while others are insoluble. Carbonates of alkali metals (e.g. Na and K) are soluble while those of alkaline earth metals are insoluble (e.g. Ca and Mg). When \ce{CO2} is in excess, the soluble hydrogen carbonate forms.
$$\ce{CO2 (g) + Ca(OH)2 (aq)$\longrightarrow$ 2CaCO3 (s) + H2O (l)}$$
and, in excess: $$\ce{CaCO3(s) + H2O(l) +  CO2(g) $\longrightarrow$ Ca(HCO3)2(aq)}$$

\subsubsection*{Materials}
clean drinking straw, lime water*, beaker*, test tube*

\subsubsection*{Activity Procedure}
\begin{enumerate}
\item{Put approximately 3 mL of clear lime water solution into a beaker.}
\item{Blow exhaled air from the mouth into the solution in the test tube and note the changes. Be careful not to suck the solution into the mouth. If the solution is sucked into the mouth rinse with clean drinking water.}
\end{enumerate}

\subsubsection*{Results and Conclusion}
Carbon dioxide blown from the straw will react with lime water \ce{Ca(OH)2} to form a white precipitate of \ce{CaCO3}. This shows that carbon dioxide reacts with an alkali solution.

\subsubsection*{Clean Up Procedure}
\begin{enumerate}
\item{Collect all the used materials, cleaning and storing items that will be used later. No special waste disposal is required.}
\end{enumerate}

\subsubsection*{Notes}
Hydrogen carbonates of alkaline earth metals can only exist together as ions solution - they cannot be prepared as solids. Outside of solution the hydrogen carbonate immediate decomposes to carbonate.

\section{Preparation of Metal Oxides}

There are several methods of producing metal oxides. This section considers a direct method -- combination with air at high temperature -- and an indirect method -- reaction with acid followed by thermal decomposition of the resulting salt.

Both of these activities may be performed by students. The number of groups will probably be limited by the number of heat sources, although adequate supervision is also a concern.

\subsection{Direct Preparation of a Metal Oxide}

\subsubsection*{Learning Objectives}
\begin{itemize}
\item{To prepare an oxide of a metal by the direct method of using heat.}
\end{itemize}

\subsubsection*{Materials}
Copper metal wire*, source of heat*, match box, spoon.

\subsubsection*{Hazards and Safety}
\begin{itemize}
\item{This experiment involves open flames a bucket of water or sand should be available for fire fighting purposes.}
\end{itemize}
\subsubsection*{Preparation}
\begin{enumerate}
\item Clean the piece of copper metal using dilute acid and/or sand paper to make sure it is a nice shiny copper colour.
\end{enumerate}
\subsubsection*{Activity Procedure}
\begin{enumerate}
\item{Take a small piece of copper and put in in the flame.}
\item{Heat the copper strongly observe the changes.}
\end{enumerate}

\subsubsection*{Results and Conclusion}
Copper metal reacts with air to form black copper (II) oxide.

\subsubsection*{Clean Up Procedure}
\begin{enumerate}
\item{Collect all the used materials, cleaning and storing items that will be used later. No special waste disposal is required.}
\end{enumerate}

\subsubsection*{Discussion Questions}
\begin{enumerate}
\item{What happened when the copper was heated strongly?}
\item{What other metal oxides can be prepared by direct method?}
\end{enumerate}
\subsection*{Notes}
This experiment can also be tried with zinc meta. Zinc metal turns yellow when heated strongly in air. The yellow colour turns white when allowed to cool. This product which is yellow when hot and white when cold is zinc oxide (ZnO).

\subsection{Indirect Preparation of a Metal Oxide}

\subsubsection*{Learning Objectives}
\begin{itemize}
\item{SWBAT prepare a metal oxide by the reaction of metal with acid followed by heating.}
\end{itemize}

\subsubsection*{Materials}
Zinc metal*, battery acid (5~M sulphuric acid), beaker*, heating vessel*, heat source*, spoon.

\subsubsection*{Hazards and Safety}
\begin{itemize}
\item{((battery acid))}
\end{itemize}

\subsubsection*{Activity Procedure}
\begin{enumerate}
\item{Put about 10 mL of sulphuric acid into a beaker.}
\item{Add a small piece of zinc metal and then allow the reaction to take place.}
\item{After the zinc granule has completely dissolved, take the solution and pour 3 mL of it into a heating vessel.}
\item{Heat the solution to dryness to obtain the residue formed.}
\item{Put small amount of this solid into a spoon and heat it until the compound decomposes and a new residue forms.}
\end{enumerate}

\subsubsection*{Results and Conclusion}
When zinc reacts with dilute sulphuric acid, a soluble zinc sulphate salt forms. This is formed by displacement reaction. The salt can be obtained only by evaporating water to dryness. ZnSO4 is white in colour. When heating the compound (ZnSO4) the gas SO2 is evolved and the residue is ZnO. The ZnO is yellow when hot and white when cold.

\subsubsection*{Clean Up Procedure}
\begin{enumerate}
\item{Collect all the used materials, cleaning and storing items that will be used later. No special waste disposal is required.}
\end{enumerate}

\subsubsection*{Discussion Questions}
\begin{enumerate}
\item{Write the equation and name the compound formed when zinc is put into acid and heated to dryness.}
\item{What is the purpose of heating the solution to dryness?}
\item{When heating the compound formed after dryness explain the changes in the residue.}
\end{enumerate}

\section{Preparation of Metal Hydroxides}

The direct preparation of a metal hydroxide is to put a reactive metal in water where it forms metal hydroxide and hydrogen gas. No locally available metals perform this reaction at a reasonable rate at room temperature. However, metal hydroxides may be prepared by an indirect method: reaction with acid followed by precipitation with sodium hydroxide.

This activity may be performed by students individually or in small groups. The chemicals involved are dangerous if handled improperly, so class size should be reduced to a number that may be easily supervised.

\subsubsection*{Learning Objectives}
\begin{itemize}
\item{To prepare metal hydroxide by an indirect method.}
\end{itemize}

\subsubsection*{Materials}
Steel wool, battery acid (5~M sulphuric acid)*, caustic soda (sodium hydroxide)*, filter funnel*, beakers*.

\subsubsection*{Hazards and Safety}
\begin{itemize}
\item{((battery acid))}
\end{itemize}

\subsubsection*{Preparation}
\begin{enumerate}
\item{Prepare a sodium hydroxide solution by adding 1 spoon of sodium hydroxide to 100 mL of water. Provide each group with 10 mL of solution in a beaker.}
\end{enumerate}

\subsubsection*{Activity Procedure}
\begin{enumerate}
\item{Instruct students to take a small amount of steel wool and to put it in one of the beakers.}
\item{Instruct students to add about 10 mL of strong acid.}
\item{Guide students to observe the reaction. Bubbles of hydrogen gas should form. Once students have observed the reaction, they should place the reacting beaker in a well ventilated space and not breathe the gas produced. While most is hydrogen, something else unpleasant also forms.}
\item{The reaction is finished when there are no more bubbles. If the steel wool is completely consumed, advise students to add more steel wool to allow the reaction to continue. The goal is for all of the acid to be consumed. Observe the colour of the final solution.}
\item{When the reaction is finished, instruct students to decant the contents of the reaction beaker into their beaker of sodium hydroxide. A precipitate should form immediately. Observe the colour of the precipitate.}
\item{Pour the mixture with the precipitate into the filter funnel. Leave to filter. Observe any change in colour.}
\item{Once most of the liquid has passed through the filter, remove the solid from the filter funnel and leave to dry.}
\end{enumerate}

\subsubsection*{Results and Conclusion}
The steel wool reacts with strong acid to form iron (II) solution. If sulphuric acid is used as the strong acid, the product of the first reaction will be iron (II) sulphate. This solution reacts with sodium hydroxide solution to produce a green, gelatinous precipitate of iron (II) hydroxide. On exposure to air, this precipitate oxidizes to iron (III) oxide.

\subsubsection*{Clean Up Procedure}
\begin{enumerate}
\item{Pour the wastes into the toilet or soak pit tank and clean the table.}
\end{enumerate}

\subsubsection*{Discussion Questions}
\begin{enumerate}
\item{What are the products of the first reaction?}
\item{What are the products of the second reaction?}
\item{What caused the change in colour as the precipitate is exposed to the air?}
\item{What is the chemical formula of the final product?}
\end{enumerate}

\section{Preparation of Sulphates}

Soluble metal sulphates may be prepared by the action of sulphuric acid on a metal. The following activity demonstrates this principle to produce zinc sulphate. The same process can be used with steel wool to prepare iron sulphate.

Students can perform these activities individually or in groups, although the class size must be small enough for adequate supervision as strong acids are required.

This activity may also be repeated at larger scale as a project to produce larger quantities of zinc sulphate or iron sulphate for use in qualitative analysis and other experiments.

\subsection{Preparation of Salts by Reaction of Metal with Acid}

\subsubsection*{Learning Objectives}
\begin{itemize}
\item{SWBAT prepare salts in the laboratory by the replacement reaction of hydrogen from an acid by a metal.}
\end{itemize}

\subsubsection*{Materials}
Zinc metal*, dilute sulphuric acid*, beakers*, evaporating dish*, kerosene stove and steel wool.

\subsubsection*{Hazards and Safety}
\begin{itemize}
\item{((battery acid))}
\end{itemize}

\subsubsection*{Preparation Procedure}
\begin{enumerate}
\item{Clean up the zinc metal by using steel wool and cut it into small pieces to increase surface area for the reaction.}
\end{enumerate}

\subsubsection*{Activity Procedure}
\begin{enumerate}
\item{Take zinc granules into one of the beaker followed by small amount of sulphuric acid and leave for the reaction to take place.}
\item{After all the zinc granules have reacted take the solution into the evaporating plate.}
\item{Heat the solution to evaporate to dryness and collect the remains.}
\end{enumerate}

\subsubsection*{Results and Conclusion}
Zinc reacts with the sulphuric acid and replaces hydrogen gas and form soluble zinc sulphate. The product formed from the evaporation of the zinc sulphate solution is the white solid zinc sulphate.

\subsubsection*{Clean Up Procedure}
\begin{enumerate}
\item{Collect all the used materials, cleaning and storing items that will be used later. No special waste disposal is required.}
\end{enumerate}

\subsubsection*{Discussion Questions}
\begin{enumerate}
\item{What happened when the sulphuric acid was added to the zinc granules?}
\item{What are the products of the solution resulting when the zinc granules reacted to completion? Write the word and chemical reaction equation.}
\item{Give the colour, name and chemical formula of the product formed after evaporation.}
\end{enumerate}

\subsubsection*{Notes}
Sulphuric acid reacts with zinc metal to produce hydrogen gas and zinc sulphate because zinc is more reactive than hydrogen.

%$$ \mathrm{Zn}_{(s)} + \mathrm{H}_2\mathrm{SO}_{4(aq)} \longrightarrow \matherm{ZnSO}_{4(aq)} + \mathrm{H}_{2(g)} $$

Zinc sulphate is a soluble salt; to form a solid you evaporate to dryness, either over a heat source or simply in the sun. The final substance should be white.

Note that insoluble metal sulphates (e.g. \ce{BaSO4} and \ce{PbSO4}) are not possible to prepare with locally available materials. However, students should understand the general principle of precipitation reactions from the preparation of metal carbonates above. This is just as well as both barium and lead compounds are poisonous.

\section{Preparation of Sulphides}

Metal sulphides are usually insoluble and darkly coloured solids. In the laboratory, metal sulphides are most easily prepared by the direct combination of the metal and sulphur powder at high temperature. This activity describes the preparation of copper sulphide but it is the same process as the preparation of iron sulphide described in the section on Chemical Change and Chemical Reactions.

\subsection{Investigating the Reaction of Sulphur with Metal}

\subsubsection*{Learning Objectives}
\begin{itemize}
\item{To explain the oxidizing properties of sulphur.}
\end{itemize}

\subsubsection*{Materials}
Sulphur powder*, copper wire*, two spoons, source of heat*, heavy scissors, and a match box

\subsubsection*{Hazards and Safety}
\begin{itemize}
\item{Sulphur vapours are harmful, therefore avoid inhaling while heating.}
\item{Perform this experiment in a well-ventilated room or outside.}
\end{itemize}

\subsubsection*{Preparation}
\begin{enumerate}
\item{Cut the copper wire into small pieces using the scissors.}
\end{enumerate}

\subsubsection*{Activity Procedure}
\begin{enumerate}
\item{Instruct students to put a few copper cuttings into a spoon.}
\item{Instruct students to add a small amount of sulphur to the copper cuttings and mix. The volume of sulphur should be large compared to the volume of copper because copper is much more dense.}
\item{Heat the mixture in the spoon over a flame until the mixture turns black.}
\end{enumerate}

\subsubsection*{Results and Conclusion}
Copper and sulphur reacted to form black copper sulphide.

\subsubsection*{Clean Up Procedure}
\begin{enumerate}
\item{Collect all the used materials, cleaning and storing items that will be used later. It may be necessary to clean the spoon with a piece of steel wool.}
\item{Dispose of chemical waste in the pit latrine.}
\end{enumerate}

\subsubsection*{Discussion Questions}
\begin{enumerate}
\item{Why does the mixture of copper and sulphur turn black? Write the chemical equation for this reaction.}
\item{Name the black compound formed.}
\end{enumerate}
