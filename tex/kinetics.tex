\chapter{Kinetics, Equilibrium, and Energetics}

\section{Kinetics}

Kinetics is the study of the rate of chemical reactions. Some reactions have a high rate, that is, they happen very quickly, for example the burning of paper. Other reactions happen very slowly, for example the rusting of metal. In order for a chemical reaction to happen, the molecules concerned must collide with each other. Anything that increases the frequency and intensity of these collisions will increase the rate of reaction.

Students should learn six factors that affect the rate of reaction: concentration, pressure, temperature, surface area, catalyst and light. This chapter discusses each of these factors and activities for presenting several of them.

\subsection{The Effect of Concentration of Reaction Rate}

Concentration has a positive effect on reaction rate. A higher concentration means that the molecules are more crowded and collide more often, thus increasing the rate of chemical reaction. In a solution of low concentration, molecules are less likely to collide with each other and hence have a slower rate of reaction. 

The effect of concentration is only observed in reactions that occur in solution. In gases, the effective concentration is directly related to the pressure of the gas - a higher pressure means a higher concentration means a faster rate of reaction. Because experiments with reacting gases are more difficult to set up and generally more dangerous to perform, students generally do experiments on the effect of concentration and use logic to extend their conclusions to the effect of pressure.

The following activity is useful for students to observe the effect of concentration on reaction rate. This activity should be performed by students working in small groups so that the effects may be easily seen.

\subsubsection*{Learning Objectives}
\begin{itemize}
\item{To demonstrate the effect of concentration on the rate of the reaction}
\end{itemize}

\subsubsection*{Materials}
vinegar, bicarbonate of soda (sodium hydrogen carbonate)*, water, 6 test tubes*, test tube rack*

\subsubsection*{Activity Procedure}
\begin{enumerate}
\item{In 3 test tubes labeled A, B and C put approximately 10mL, 5mL and 2.5mL of vinegar respectively. Dilute each solution to approximately 20 mL.}
\item{In 3 new test tubes labeled D, E, and F place 10g, 5g and 2.5g of sodium bicarbonate respectively. Dilute each solution to 20 mL with water.}
\item{Mix solution A with solution D and observe what happens.}
\item{Mix solution B with solution E and observe what happens.}
\item{Mix solution C with solution F and observe what happend. Note any difference between the reactions in steps 3, 4, and 5.}
\end{enumerate}

\subsubsection*{Results and Conclusion}
Bubbles (carbon dioxide gas) will be formed most quickly in the reaction between solutions A and D. The bubbles will form more slowly in the reaction solutions B and E and the reaction will be slowest in the reaction between C and F. 

This experiment shows that higher the concentration of reactants, the faster the reaction will proceed.

\subsubsection*{Clean Up Procedure}
\begin{enumerate}
\item{Collect and clean all the used materials, storing items that will be used later. No special waste disposal required.}
\end{enumerate}

\subsubsection*{Discussion Questions}
\begin{enumerate}
\item{Why would this experiment be dangerous with concentrated acetic acid? (hint: vinegar is about 6\% acetic acid)}
\item{What will happen to the rate of a chemical reaction as the reaction proceeds?}
\end{enumerate}

\subsection{Kinetics: Effect of Temperature}

An increase in temperature increases the rate of chemical reactions by increasing both the frequency of collisions between molecules and the intensity of these collisions. A higher temperature means that the molecules are moving with a higher velocity and thus collide both more often and more forcefully. Faster, harder collisions means a much faster chemical reaction.

The following activity is useful for students to observe this effect. They may reasonably perform this experiment on the same day as the experiment regarding the effect of concentration. Students should perform this experiment themselves in groups.

\subsubsection*{Learning Objectives}
\begin{itemize}
\item{To demonstrate the effect of temperature on the rate of reaction.}
\end{itemize}

\subsubsection*{Materials}
Vinegar (acetic acid)*, bicarbonate of soda (sodium hydrogen carbonate)*, beakers*, test tubes*, water, source of heat*

\subsubsection*{Preparation}
\begin{enumerate}
\item{A few minutes prior to class light the heat source and put some water on to heat.}
\item{Prepare a solution of soidium hydrogen carbonate by dissolving approximately 3 teaspoons per litre of water.}
\end{enumerate}

\subsubsection*{Activity Procedure}
\begin{enumerate}
\item{Arrange the students into groups of 4-6. To each group give 4 test tubes, a beaker containing approximately 10 mL of acid and a second beaker containing approximately 10 mL of base.}
\item{Instruct students to arrange test tubes in the rack and label them with numbers 1, 2, 3 and 4. Instruct them to put approximately 3 mL of acid into test tubes 1 and 2 and 3 mL of base into test tubes 3 and 4.}
\item{Instruct students to heat test tubes 2 and 4 in the boiling water bath until they are nearly boiling.}
\item{Instruct students to pour the solution from test tube 3 into test tube 1.}
\item{Instruct students to then pour the solution from test tube 4 into test tube 2. Have students record their observations about the differences between the two reactions.}
\end{enumerate}

\subsubsection*{Results and Conclusion}
In the reaction between test tube 2 and 4 the reaction will be notably faster than in the reaction between test tube 1 and 3. The students should see that the bubbles are formed more quickly--the reaction is more vigorous. The hot solutions will react faster than the cold soluitons.

\subsubsection*{Clean Up Procedure}
\begin{enumerate}
\item{Collect all the used materials, cleaning and storing items that will be used later. No special waste disposal is required.}
\item{Unused acid and base solutions can be stored and labelled for later use.}
\end{enumerate}

\subsubsection*{Discussion Questions}
\begin{enumerate}
\item{In which reaction did the reaction happen faster? How do you know?}
\item{Explain what effect temperature has on the rate of a chemical reaction.}
\item{Explain why it is important to keep vegetables in a cool place during the day rather than in the sun.}
\item{The human body must be kept at a constant temperature. Explain what would happen in the temperature gets to high or low.}
\end{enumerate}

\subsubsection*{Notes}
This experiment can be expanded by setting up a gas collection apparatus and recording the volume of gas collected per unit of time for each reaction.

\subsection{Kinetics: Effect of Surface Area}

The surface area of a solid can affect the rate of chemical reactions by increasing the rate of collisions between molecules. A larger surface area means that there are more molecule available at any moment for reacting thus there are more frequent collisions. This is because in a solid, only molecules on the outer surface are able to react. If the solid is divided into many smaller pieces, the total surface area increases, the frequency of collisions increases, and therefore the rate of reaction increases.

The following activity uses the reaction between iron metal and dilute sulphuric acid to allow students to observe the effect of surface area on reaction rate. These materials are available everywhere. Students may perform this activity in groups, with close supervision as the use of a strong acid is required.

In parts of the country with carbonate rocks - coral rock, limestone, and marble - an alternative activity is the reaction of calcium carbonate and a dilute weak acid. Students should react citric acid solution or ethanoic acid solution with both large pieces and powders of these rocks. A clear difference in the rate of effervesence will be observed.

\subsubsection*{Learning Objectives}
\begin{itemize}
\item{To show the effect of surface area of a reacting solid on the rate of reaction.}
\end{itemize}

\subsubsection*{Materials}
dilute sulphuric acid*, iron nail (or any other solid iron object like mbulumbulu), iron wool, test tubes*

\subsubsection*{Hazards and Safety}
\begin{itemize}
\item{((dilute strong acid))}
\end{itemize}

\subsubsection*{Activity Procedure}
\begin{enumerate}
\item{Fill two test tubes half way with dilute sulphuric acid.}
\item{At the same time, put a nail in one test tube and a piece of steel wool into the other.}
\item{In both test tubes, bubbles of hydrogen gas should be observed on the iron. The rate of bubble formation, however, should be much faster on the steel wool. After a minute, the difference in the rate of reaction should be observed.}
\end{enumerate}

\subsubsection*{Results and Conclusion}
When the iron is placed in the acid, bubbles of hydrogen gas should clearly be seen on the surface. The bubbles form much more quickly from the steel wool than from the iron nail because it has a much higher surface area.

\subsubsection*{Clean Up Procedure}
\begin{enumerate}
\item{Collect the solutions of acid from each group. Use tweezers to remove the iron metal pieces, rinse them well in water and store for later use.}
\item{Store the sulphuric acid in a bottle labelled "Impure sulphuric acid" and save for later use. If no bottle is available, neutralise the acid with baking powder until effervescence stops and then dispose down the drain.}
\item{Collect all the used materials, cleaning and storing items that will be used later.}
\end{enumerate}

\subsubsection*{Discussion Questions}
\begin{enumerate}
\item{Which object has a higher surface area, a nail or a piece of steel wool?}
\item{In which test tube did the reaction occur faster? How do you know?}
\item{Why do people usually grind salt before using it to cook?}
\end{enumerate}

\subsubsection*{Notes}
Different brands of steel wool are more or less effective for this reaction. This is because some are coated with a material that decreases the rate of reaction (to slow down corrosion). It is best to choose a brand of steel wool that you know rusts quickly - it is probably not coated and thus better for comparison. Likewise, choose a nail that is not galvanized. Galvanization coats a nail with a layer of zinc to prevent corrosion. To make sure you have a nail that is iron, find one with a little bit of rust and then remove the rust with sand paper or steel wool.

\subsection{Effect of catalyst on reaction rate}

A catalyst is any substance that increases the rate of a chemical reaction without being consumed in the reaction. There are many different catalysts and they work in many different ways. One simple example of a catalyst is the effect of manganese (IV) oxide on the decomposition of hydrogen peroxide. This is the same experiment described above for the preparation of oxygen. Hydrogen peroxide decomposes on its own to form water and oxygen, but slowly. In the presence of manganese (IV) oxide, the decomposition is much faster.

Safety: manganese (IV) oxide is poisonous if it enters the body. While manganese (IV) oxide cannot pass through skin, it should not be handled by anyone with open cuts on his or her fingers or hands. Care should be taken when using manganese (IV) oxide to minimize the amount of manganese (IV) oxide that touches skin – careful use of tools will prevent contact altogether. Anyone using manganese (IV) oxide should thoroughly wash his or her hands with soap after this experiment. In addition, manganese (IV) oxide will corrode metal over time. Wash all metals tools that touch manganese (IV) oxide after use. Make sure all manganese (IV) oxide is removed.

Note that a substance is a catalyst only if it increases the rate of reaction. A substance that decreases the rate of reaction is called an inhibitor.

\section{Equilibrium}

Equilibrium is a state of balance reached when the forward and reverse process have the same rate. In chemistry, equilibrium applies to reversible reactions. The easiest examples of a reversible reaction involve acid-base indicators. Simply fill a beaker with a dilute solution of a colourful indicator and alternate adding acid and base. The colour will shift back in forth with each addition.

The following activity is useful for demonstrating the difference between reversible and irreversible chemical reactions. Students may perform this activity themselves, either individually or in very small groups.

\subsection{Reversible Chemical Reaction}

Copper (II) sulphate exists in two forms: hydrated (with water) and anhydrous (without water). Hydrated copper (II) sulphate is blue while anhydrous copper (II) sulphate is white. In this activity, students dehydrate and then rehydrate copper (II) sulphate to observe a reversible reaction. To show an irreversible chemical reaction, use one of the reactions from the thermal decomposition activity of chemical reactions, that is heating copper carbonate or citric acid until only a black residue remains.

\subsubsection*{Learning Objectives}
\begin{itemize}
\item{To demonstrate a reversible reaction}
\end{itemize}

\subsubsection*{Materials}
Heat source*, copper (II) sulphate*, water, long metal spoon, dropper, sand

\subsubsection*{Hazards and Safety}
\begin{itemize}
\item{This activity requires prolonged heating with a metal spoon. Use a long spoon and hold at the end to prevent burns. Also be very careful with the spoon at the end of the experiment - it will be very hot!}
\end{itemize}

\subsubsection*{Preparation}
\begin{enumerate}
\item{Grind the copper (II) sulphate crystals into a powder. If the powder looks white, leave it exposed to the air until it regains a blue colour.}
\end{enumerate}

\subsubsection*{Activity Procedure}
\begin{enumerate}
\item{Make a small pile of sand on the bench, about 3 cm in diameter.}
\item{Place a very small amount of blue copper (II) sulphate in a metal spoon. The amount should not exceed half a centimeter in diameter.}
\item{Heat the spoon gently over a heat source. Stop heating when the crystals have changed from blue to white.}
\item{Rest the metal spoon on the pile of sand. Leave it for one minute to cool.}
\item{Add 1 drop of water to the white crystals. If no change is observe, continue adding water drop by drop until a change is seen.}
\item{Observe and record any colour change.}
\item{Rest the metal spoon on the pile of sand. Leave it for five minutes to cool.}
\end{enumerate}

\subsubsection*{Results and Conclusion}
On heating blue hydrated copper (II) sulphate, the colour changes from blue (CuSO4-5H2O) to white (CuSO4). On addition of a few drops of water, CuSO4 returns to its original hydrated state (blue).

\subsubsection*{Clean Up Procedure}
\begin{enumerate}
\item{Copper (II) sulphate crystals can be left in the air to dry the excess water and then used again in future experiments.}
\item{Collect all the used material, cleaning and storing items that will be used again later. No special waste disposal is required.}
\end{enumerate}

\subsubsection*{Discussion Questions}
\begin{enumerate}
\item{Copper sulphate can be used as a test for water. Explain how this is possible.}
\item{Give an example of another reversible reaction that you have seen before.}
\end{enumerate}

%\subsubsection*{Notes}
%The clever student may note that the conversion of copper (II) sulphate to copper (II) oxide is in fact reversible. This student should be congratulated - copper (II) oxide can indeed be converted to copper (II) sulphate by the addition of a few drops of sulphuric acid followed by gentle heating. In this manner, all reactions are in fact `reversible' - even the products of combustion can be collected and processed to remake the original substance. Explain that here reversible means directly reversible - in this case water was removed (by heating) and then returned (by dropper). The exact reaction was reversed. In the case of copper (II) oxide, different reactions are required to convert the products back to the reactants. Only reactions that are directly reversible can form an equilibrium.

\section{Energetics}

Many chemical reactions cause a change in temperature. In these reactions, energy is not being created nor destroyed, but converted from one form to another. Specifically, energy is converted between thermal energy and chemical energy. Thermal energy is the movement of molecules -- this is measured by the temperature. Chemical energy is a kind of potential energy molecules have for reaction. Very reactive chemicals have a high potential energy. Very stable chemicals have a low potential energy.

In many reactions, reactive chemicals react to form more stable chemicals. In this case the amount of potential energy held by the chemicals decreases. This energy is converted to thermal energy, causing the temperature to increase. Such a reaction is called exothermic.

In some reactions, the potential energy of the reacting chemicals increases. This energy is taken from thermal energy. Such reactions are called endothermic -- thermal energy is converted to chemical potential energy and the temperature decreases.

Note that the words `exothermic' and `endothermic' are often defined with regard to the transfer of heat to the environment. This definition confuses many students. For example, an exothermic reaction transfers heat to the environment. Many students therefore expect the temperature of their reacting mixture to decrease, because heat should be transferred away from the reaction and out into the environment. Of course, this is incorrect. In an exothermic reaction, first, chemical potential energy is transformed into thermal energy, and then that thermal energy moves into the environment. The temperature of the reaction mixture will quickly increase as heat is created by the reaction and then slowly decrease back to room temperature.

\subsection{Endothermic and Exothermic Reactions}

The ideas of endothermic and exothermic reactions are easiest to understand with examples dramatic enough to feel the difference. The following activity may be used to allow students to feel the temperature change for both an endothermic and an exothermic reaction. Students should therefore perform this activity in the smallest possible groups to ensure each can feel the change in temperature.

\subsubsection*{Learning Objectives}
\begin{itemize}
\item{To demonstrate an exothermic reaction and an endothermic reaction}
\end{itemize}

\subsubsection*{Materials}
3 small plastic bottles with caps, spoons, powdered laundry soap (e.g. Omo, Foma), calcium ammonium nitrate* OR citric acid 

\subsubsection*{Activity Procedure}
\begin{enumerate}
\item{Add about 100 mL of water to each bottle.}
\item{Add 3 spoons of powdered laundry soap to the first bottle. Cap the bottle and shake vigorously. Feel the bottle to observe any change in temperature.}
\item{Add 3 spoons of calcium ammonium nitrate or citric acid to the second bottle. Cap the bottle and shake vigorously. Feel the bottle to observe any change in temperature.}
\item{Add nothing to the third bottle and shake vigorously. Compare the temperature of this bottle with the two previous bottles.}
\end{enumerate}

\subsubsection*{Results and Conclusion}
The dissolution of laundry soap in water is exothermic -- this bottle should get warmer. The dissolution of calcium ammonium nitrate and citric acid are both endothermic -- this bottle should get cooler.

\subsubsection*{Clean Up Procedure}
\begin{enumerate}
\item{Collect the calcium ammonium nitrate or citric acid solutions into one large container. Label this container ``calcium ammonium nitrate solution, 3 spoons / 100 mL'' or ``citric acid solution, 3 spoons / 100 mL'' as appropriate. Calcium ammonium nitrate solution can be saved for producing ammonia or may be diluted in a much larger quantity of water and used to fertilize crops. Citric acid solution should remain in the lab for use as a general weak acid.}
\item{Label the bottles of laundry soap solution "soap." Use a hammer and nail or a heated nail to makes holes in the lids of these bottles. Keep some of these bottles for washing hands and apparatus in the lab and put the others in the school latrines for students to wash their hands.}
\end{enumerate}

\subsubsection*{Discussion Questions}
\begin{enumerate}
\item{What are other examples of exothermic chemical reactions? A good answer is ``fire!''}
\end{enumerate}

\subsubsection*{Notes}
Another example of an exothermic reaction is the addition of caustic soda (sodium hydroxide) to water. If you need to make sodium hydroxide solution for an experiment, prepare it yourself in front of the students and then pass the capped bottle around for them to feel. Sodium hydroxide is corrosive and therefore dangerous -- do not use it for this simple experiment; laundry soap will work just as well.