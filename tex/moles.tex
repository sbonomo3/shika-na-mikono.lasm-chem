\chapter{Moles}

\subsection{A Mole of Water}

\subsubsection*{Learning Objectives}
\begin{itemize}
\item{To measure a mole of water}
\end{itemize}

\subsubsection*{Materials}
plastic syringe, water, beaker*

\subsubsection*{Activity Procedure}
\begin{enumerate}
\item{Use the syringe to transfer 18 mL of water to the beaker.}
\end{enumerate}

\subsubsection*{Discussion Questions}
\begin{enumerate}
\item{What mass of water is in the beaker?}
\item{How many moles of water are in the beaker?}
\end{enumerate}

\subsubsection*{Notes}
The density of water at room temperature is approximately 1 g/mL. Therefore, 18 mL of water is 18 g of water. The molecular mass of water is 18 g / mol. Therefore 18 g of water is one mole of water. In the beaker now is one mole of water. Compare this volume to the volume of one mole of gas (next activity!)
\begin{center}
NB: 1 mL = 1 cm $^{3}$ = 1 cc
\end{center}

\subsection{A Mole of Gas}

\subsubsection*{Learning Objectives}
\begin{itemize}
\item{To construct a box holding a mole of gas}
\end{itemize}

\subsubsection*{Materials}
Card board, knife, ruler, masking tape and super glue.

\subsubsection*{Hazards and Safety}
\begin{itemize}
\item{Be careful with the knife and glue to avoid accidents.}
\end{itemize}

\subsubsection*{Preparation}
\begin{enumerate}
\item{Collect all the needed materials and put them on the bench.}
\end{enumerate}

\subsubsection*{Activity Procedure}
\begin{enumerate}
\item{Draw three side of a box each having 28.2~cm long.}
\item{Cut the pieces to obtain equal size.}
\item{Fold the edges to obtain the cube.}
\item{Bind the edges of the box with glue or masking tape.}
\end{enumerate}

\subsubsection*{Results and Conclusion}
The volume of the box is the product of the length, width and height. \\(ie: $28.2 cm \times 28.2 cm \times 28.2cm = 22425.768cm^{3} = 22.4dm^{3}$).

\subsubsection*{Clean Up Procedure}
\begin{enumerate}
\item{Remove all the unwanted materials and dispose of them safely.}
\end{enumerate}

\subsubsection*{Discussion Questions}
\begin{enumerate}
\item{Calculate the volume of the box in cubic centimetres.}
\item{Convert the volume obtained into cubic decimetres.}
\end{enumerate}

\subsubsection*{Notes}
The volume of any gas at s.t.p contains the volume equal to the volume of this box, ie: 22.4~$dm^{3}$. This is called gramme molecular volume(G.M.V.) when the molecular weight of a given gas is expressed in grams. Eg: 2g of H$_{2}$(g), 32g of O$_{2}$(g), 17g of NH$_{3}$(g) and 44g of CO$_{2}$(g).
