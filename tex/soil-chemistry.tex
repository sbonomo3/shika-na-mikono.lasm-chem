\chapter{Soil Chemistry}

The study of soil chemistry covers four topics for practical work. First, students learn about the formation of soil, particularly the effect of chemical weathering. Second, students learn how to measure the pH of soils to know if the soil is excessively acidic or basic. Third, students learn how to both raise and lower the pH of soils, either by liming or the addition of ammonium sulphate or urea. Finally, students study the leaching process.

\subsection{Soil Formation}

Soils are made by physical and chemical weathering of rocks. Chemical weathering is caused by the action of acids on carbonate rocks. These acids may be organic acids produced by soil organisms or it might be carbonic acid from the dissolution of atmospheric carbon dioxide in water:

\begin{center}
\ce{CO2$_{(g)}$ $\longleftrightarrow$ CO2$_{(aq)}$}
\end{center}

and then

\begin{center}
\ce{CO2$_{(aq)}$ + H2O$_{(l)}$ $\longrightarrow$ H2CO3$_{(aq)}$ $\longleftrightarrow$ H+$_{(aq)}$ + HCO3-$_{(aq)}$}
\end{center}

These acids react differently with different kinds of rocks. In the case of limestone, marble, coral, and other rocks made mostly from calcium carbonate, the reaction is:

\begin{center}
\ce{2H+$_{(aq)}$ + CaCO3$_{(s)}$ $\longrightarrow$ Ca2+$_{(aq)}$ + 2HCO3-$_{(aq)}$}
\end{center}

Note that the result is a solution of calcium hydrogen carbonate. This is the source of hard water.

Chemical weathering is a slow process. This activity speeds up the process by using dilute sulphuric acid so that students may more quickly see the result. Because sulphuric acid is a strong acid, it will also react with the hydrogen carbonate:

\begin{center}
\ce{H+$_{(aq)}$ + HCO3-$_{(aq)}$ $\longrightarrow$ H2O$_{(l)}$ + CO2$_{(g)}$}
\end{center}

The carbon dioxide thus produced can be observed as small bubbles forming on the surface of the rock. Note that in real chemical weathering, carbon dioxide is generally not produced, and instead the reaction stops with a solution of calcium hydrogen carbonate.

\subsubsection{Objective}
\begin{itemize}
\item{To understand chemical weathering by observing the effect of acids on carbonate rock}
\end{itemize}

\subsubsection{Materials}
Dilute sulphuric acid*, calcium carbonate rock (coral, limestone, marble) or egg shells, beaker*

\subsubsection{Activity Procedure}
\begin{enumerate}
\item{Place carbonate rock or egg shells in a beaker.}
\item{Add dilute sulphuric acid. Make observations}
\end{enumerate}

\subsubsection{Results and Conclusion}
Bubbles of carbon dioxide should be observed where acid touches the rock. This shows that the acid is chemically reacting with the rock. Over time, the surface of the rock should also look corroded, more evidence of chemical weathering.

\subsubsection{Clean-Up}
\begin{enumerate}
\item{Save the rock and unused dilute sulphuric acid for future experiments.}
\item{Neutralize any acid waste prior to disposal.}
\end{enumerate}

\subsubsection{Notes}
Rocks with more complicated chemical composition are also subject to chemical weathering. Their reactions are more complicated, and something students will study if they pursue soil science or geology at the University.

%\subsection{Measuring Soil pH}

%Some soils are neutral in pH. Others are acidic or basic, depending on the composition of the soil. This activity is meant to demonstrate the existence of acidic and basic soils. Traditionally, this activity is performed with universal indicator. However, exceptionally acidic or basic soils should be possible to detect using red and blue indicating paper, which may be locally made.

%\subsubsection{Objective}
%\begin{itemize}
%\item{To measure the acidity or basicity of soil}
%\end{itemize}

%\subsubsection{Materials}
%Various soil samples, red and blue indicating paper (from the Acids and Bases section), beakers*, water

%\subsubsection{Activity Procedure}
%\begin{enumerate}
%\item{Put soil in a beaker.}
%\item{Add water to the soil and stir.}
%\item{Test the liquid with red and blue indicating paper. Record any changes.}
%\end{enumerate}

%\subsection{Raising Soil pH by Liming}

%\subsubsection{Objective}
%\begin{itemize}
%\item{To demonstrate the ability of lime to increase the pH of soil}
%\end{itemize}

%\subsubsection{Materials}
%An acidic soil from the above activity, lime (cement or calcium hydroxide)*, water, red and blue indicator paper*

%\subsubsection{Activity Procedure}
%\begin{enumerate}
%\item{To an acidic soil sample from the soil pH activity, add lime.}
%\item{Test again with indicator paper.}
%\end{enumerate}

%\subsection{Lowering Soil pH with Ammonium Sulphate}

%\subsubsection{Objective}
%\begin{itemize}
%\item{To demonstrate the ability of ammonium sulphate to decrease the pH of soil}
%\end{itemize}

%\subsubsection{Materials}
%A basic soil from the soil pH activity, ammonium sulphate fertilizer, water, red and blue indicating paper*

%\subsubsection{Activity Procedure}
%\begin{enumerate}
%\item{To a basic soil sample from the soil pH activity, add ammonium sulphate.}
%\item{Test again with indicator paper.}
%\end{enumerate}

\subsection{Leaching}

\subsubsection{Objective}
\begin{itemize}
\item{To demonstrate the effect of leaching}
\end{itemize}

\subsubsection{Materials}
Sand, solid food colouring, filter funnel*, beaker*, water.

\subsubsection{Activity Procedure}
\begin{enumerate}
\item{Prepare a mixture of sand and solid food colouring.}
\item{Emphasize that the food colouring represents soluble minerals and soil nutrients.}
\item{Place the sand mixture in a filter funnel placed over a beaker.}
\item{Add water and observe the colour of the filtrate.}
\item{Discuss the manner in which soluble soil nutrients are leached.}
\end{enumerate}