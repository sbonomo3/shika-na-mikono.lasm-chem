\chapter{Water}

Water is fundamental to life. It is also essential in the chemistry laboratory. This section includes a series of activities to teach students about water - how to test for it, how to purify it, and much about hard and soft water.

\subsection{Test for Water}

\subsubsection*{Learning Objectives}
\begin{itemize}
\item{To understand the meaning of the words `hydrated' and `anhydrous.'}
\item{To test for the presence of water.}
\end{itemize}

\subsubsection*{Background Information}
Copper (II) sulphate exists in two forms: hydrated (with water) and anhydrous (without water). Hydrated copper (II) sulphate is blue while anhydrous copper (II) sulphate is white. Thus, white copper sulphate can be used as a test for water.

\subsubsection*{Materials}
Heat source*, copper (II) sulphate*, water, metal spoon

\subsubsection*{Hazards and Safety}
\begin{itemize}
\item{This experiment involves open flames a bucket of water or sand should be available for fire fighting purposes.}
\end{itemize}

\subsubsection*{Activity Procedure}
\begin{enumerate}
\item{Place a very small amount of blue copper (II) sulphate in a metal spoon.}
\item{Heat the spoon gently over the heat source. Stop heating when the crystals have changed from blue to white.}
\item{Add a few drops of water to the white crystals. Observe and record any colour change.}
\end{enumerate}

\subsubsection*{Results and Conclusion}
On heating blue hydrated copper (II) sulphate, the colour changes from blue (\ce{CuSO4$\cdot$5H2O}) to white (\ce{CuSO4}). On addition of a few drops of water \ce{CuSO4} returns to its original hydrated state (blue), i.e. copper sulphate pentahydrated.

\subsubsection*{Clean Up Procedure}
\begin{enumerate}
\item{Copper (II) sulphate crystals can be left in the air to dry the excess water and then used again in future experiments.}
\item{Collect all the used material, cleaning and storing items that will be used again later. No special waste disposal is required.}
\end{enumerate}

\subsubsection*{Discussion Questions}
\begin{enumerate}
\item{Explain what happens to the blue copper (II) sulphate when it is heated? What happened when water was added?}
\item{Write a balanced chemical equation for this reaction.}
\item{When anhydrous copper (II) sulphate is allowed to sit out for 30 minutes its colour changes from white to light blue. Explain why this happens.}
\end{enumerate}

\subsubsection*{Notes}
The balanced chemical equation for this reaction is 
\begin{center}
\ce{CuSO4}$\leftrightharpoons$ \ce{CuSO4$\cdot$ 5H2O}
\end{center}

\subsection{Water Purification}

Natural water may contain harmful organisms and substances. Water purification is necessary before drinking in order to removed harmful micro-organisms and dirt particles. Water treatment is the process of making water safe and usable for either domestic, industrial or medical purpose. This activity allows students to practice proper water treatment.



\subsubsection*{Objectives}
\begin{itemize}
\item{To demonstrate proper treatment of drinking water.}
\end{itemize}

\subsubsection*{Materials}
clean piece of white cloth, boiling vessel (sufuria), heat source, impure water, bucket

\subsubsection*{Hazards and Safety}
\begin{itemize}
\item{This experiment involves open flames a bucket of water or sand should be available for fire fighting purposes.}
\end{itemize}


\subsubsection*{Activity Procedure}
\begin{enumerate}
\item{Make the source of heat by lighting the stove.}
\item{Heat the water to boiling. Allow water to boil for 5 minutes.}
\item{Allow water to cool.}
\item{Use the clean white cloth to filter the water into a clean bucket. Water is now safe for drinking.}
\end{enumerate}

\subsubsection*{Results and Conclusion}
The act of boiling water at the boiling point (100˚C) kills the germs and bacteria which may cause disease. Filtration with a piece of clean white cloth removes any solid impurities. The filtrate obtained is drinkable water.

\subsubsection*{Discussion Questions}
\begin{enumerate}
\item{Define the following terms (a) water treatment (b) water purification.}
\item{It is advised to let water boil for at least 5 minutes. Why?}
\item{Explain how the filtered water is different from the original water.}
\end{enumerate}

\subsubsection*{Notes}
Instead of a cloth, it is possible to use chemical purifiers for filtration such as ``water guard" or ``aqua guard".

\subsection{Differences between Soft Water and Hard Water}

The names soft water and hard water were developed before people understood chemistry. Even today the average person is familiar with the difference between water from place to place. In some places it is easy to use soap - it forms a soft lather easily and cleans hands and clothes effectively. In these places the water is called soft water. In other places, it is hard to use soap - much effort is required to get the lather required for cleaning. In these places the water is said to be hard water.

Now we know what causes some water to be hard water and other water to be soft water. Hard water contains dissolved magnesium or calcium ions. Soft water does not. While we cannot feel the difference between plain soft water and plain hard water, the difference is clear when using soap. Dissolved magnesium and calcium ions bind to the soap, forming a `scum' that is not effective for cleaning. Only with much soap can a useful lather be formed.

The following activity is useful for students to experience the difference between soft water and hard water. Students should perform this activity in small groups.

\subsubsection*{Objectives}
\begin{itemize}
\item{To differentiate between soft water and hard water.}
\end{itemize}

\subsubsection*{Materials}
Soft water (rain/distilled water), gypsum powder (calcium sulphate), Epsom salt (magnesium sulphate), table salt, test tubes*, and a bar of soap (or liquid soap)

\subsubsection*{Preparation}
\begin{enumerate}
\item {Prepare a sample of magnesium sulphate by dissolving 1 table spoon of Epsom salts per 100~mL of water.}
\item {Prepare a sample of calcium sulphate by dissolving 1 tablespoon of gypsum powder per 100~mL of water and stirring. Decant off the solution, leaving behind any undissolved gypsum}
\item {Prepare a sample of sodium chloride by dissolving 1 table spoon of table salt per 100~mL of water.}

\end{enumerate}

\subsubsection*{Activity Procedure}
\begin{enumerate}
\item{Label 5 test tubes: (1) \ce{CaSO4} solution, (2) \ce{MgSO4} solution, (3) \ce{NaCl} solution, (4) \ce{MgSO4 / CaSO4} solution and (5) soft water.}
\item{In each beaker put 4-5~mL of the respective solution.}
\item{Put a few small pieces of soap (or drops if liquid soap is used) in each test tube and shake vigorously.}
\item{Record the observations about each beaker.}
\end{enumerate}

\subsubsection*{Results and Conclusion}
The three beakers labelled \ce{CaSO4} solution, \ce{MgSO4} solution, and \ce{MgSO4 / CaSO4} solution all form soap scum and a small amount of lather, while the other two beakers containing soft water and NaCl solution easily formed a lather with soap.

\subsubsection*{Clean Up}
\begin{enumerate}
\item{Collect all the used materials, cleaning and storing items that will be used later. No special waste disposal is required.}
\end{enumerate}

\subsubsection*{Discussion Questions}
\begin{enumerate}
\item{What causes hardness of water?}
\item{How can you differentiate hard water from soft water?}
\end{enumerate}

\subsubsection*{Notes}
Permanent hardness of water is caused by the presence of Ca$^{+2}$ and Mg$^{+2}$ ions in solution. The two ions react with soap to form insoluble substance called \textit{scum}. The formation of scum destroys the soap and prevents the formation of lather. 
\begin{center}
Ca or Mg salt (aq) + soap (aq)  $\longrightarrow$ scum (s) + sodium salt (aq)\\
\ce{CaSO4(aq) + 2NaSt(aq) $\longrightarrow$ CaSt(s) + Na2SO4}\\
\ce{MgSO4(aq) + 2NaSt(aq) $\longrightarrow$ MgSt(s) + Na2SO4}\\
\end{center}
(St represents the stearate ion \ce{CH3(CH2)16COO^{-}} in these chemical equations.)

\subsection{Temporary Hard Water}

There are two kinds of hard water: temporary hard water and permanent hard water. A long time ago, people thought that only temporary hard water could be treated, whereas permanent hard water would always be hard. Now we know how to treat both kinds of hard water.

Temporary hard water has hydrogen carbonate anions. When temporary hard water is boiled, the hydrogen carbonate decomposes to form carbonate. Neither magnesium carbonate nor calcuim carbonate is soluble in water, so these salts precipitate. Therefore temporary hard water can be boiled to produce soft water.

The following activity is useful for students to observe the treatment of temporary hard water by boiling. Students should perform this activity in small groups, one for each available heat source.

\subsubsection*{Objectives}
\begin{itemize}
\item{To treat hard water by heating.}
\end{itemize}

\subsubsection*{Materials}
beakers*, test tubes*, soap, heating vessel*, Epsom salt (magnesium sulphate), bicarbonate of soda (sodium hydrogen carbonate), heat source*, match box, soft water (rain/distilled water), filter paper*, funnel* and washing soda {sodium carbonate)*.

\subsubsection*{Hazards and Safety}
\begin{itemize}
\item{This experiment involves open flames a bucket of water or sand should be available for fire fighting purposes.}
\end{itemize}

\subsubsection*{Preparation}
\begin{enumerate}
\item{In a beaker labelled ``temporary hard water" put one tablespoon of magnesium sulphate salt followed by two tablespoons  of sodium hydrogen carbonate and water up to a height of 3-4 cm and stir well to mix making sure that all of the salts dissolve.}
\item{In a second beaker labelled ``permanent hard water" add one tablespoon of magnesium sulphate and fill with water up to a height of 3-4 cm and stir well until all salt is dissolved.}
\item{In a third beaker labelled ``soft water"  add soft rain water up to 3-4 cm.}
\item{Prepare a sample of sodium carbonate solution by dissolving approximately 2 tablespoons of washing soda (\ce{Na2CO3}) into 500~mL of water}
\end{enumerate}

\subsubsection*{Activity Procedure}
\begin{enumerate}
\item{In three separate test tubes put approximately 5~mL samples of temporary hard water, permanent hard water and soft water}
\item{To each test tube add a small piece of soap and shake vigorously. Record your observations.}
\item{In three separate test tubes put approximately 5~mL samples of temporary hard water, permanent hard water and soft water}
\item{To each test tube add a few drops of sodium carbonate solution. Record your observations.}
\item{Take a small amount (about 5-10 mL) of temporary hard water and boil until no more precipitate is formed. Remove from heat and let it cool.}
\item{Take a small amount (about 5-10 mL) of permanent hard water and boil for a few minutes. Remove from heat and let it cool.}
\item{Put 3-5~mL of boiled permanent hard water in a clean test tube. Add a few drops of sodium carbonate and record observations.}
\item{Filter the temporary hard water solution which has formed a precipitate upon boiling and collect the clear filtrate in a clean test tube.}
\item{Put a few drops of sodium carbonate into the filtrate. Record observations.}
\end{enumerate}

\subsubsection*{Results and Conclusion}
Magnesium sulphate mixes with sodium hydrogen carbonate solution to produce  an aqueous solution of magnesium hydrogen carbonate (temporary hard water). Magnesum sulphate solution is permanent hard water. When the temporary hard water is boiled, a white precipitate of magnesium carbonate should be observed. Boiling the permanent hard water will not cause precipitation.
The precipitation occurs because the hydrogen carbonate decomposes on boiling to form a carbonate which then precipitates with magnesium ions:
\begin{center}
On boiling: \ce{2HCO3- (aq) $\longrightarrow$ H2O(l) + CO2(g) + CO3 ^{2-}(aq)}\\

\ce{CO3^{2-}(aq) + Mg2+(aq) $\longrightarrow$ MgCO3(s)}
\end{center}
After boiling and filtering, the sodium carbonate can indicate the presence or lack of magnesium ions. In the soft water and the boiled, filtered temporary hard water, no precipitate will be observed because there is no magnesium ion is present. In the permanent hard water, a precipitate will be observed with sodium carbonate


\subsubsection*{Clean Up Procedure}
\begin{enumerate}
\item{Collect all the used materials, cleaning and storing items that will be used later. No special waste disposal is required.}
\end{enumerate}

\subsubsection*{Discussion Questions}
\begin{enumerate}
\item{Define temporary hardness of water.}
\item{Why were magnesium sulphate and sodium hydrogen carbonate mixed? Write the chemical formulae of the resulting solution.}
\item{What is the precipitate formed during boiling? Write a balanced chemical equation for this reaction.}
\item{Describe why there was precipitate in only one beaker.}
\item{What are other ways of removing hardness of water?}
\end{enumerate}

\subsubsection*{Notes}
Temporary hard water is often described as containing calcium hydrogen carbonate and magnesium hydrogen carbonate. This is a source of confusion. Temporary hard water indeed contains calcium/magnesium ions as well as hydrogen carbonate ions and thus can correctly be called a solution of calcium hydrogen carbonate (or magnesium hydrogen carbonate). Note, however, that neither calcium hydrogen carbonate nor magnesium hydrogen carbonate exists as a solid chemical. Thus the preparation of temporary hard water in the laboratory is generally accomplished by adding calcium and magnesium with one salt and hydrogen carbonate with another.

\subsection{Treating Hard Water by Precipitation}

Permanent hard water contains magnesium and calcium ions without hydrogen carbonate ions. Generally the charge of the magnesium and calcium ions is balanced with chlorides and/or sulphates. Because there are no hydrogen carbonate ions, boiling permanent hard water has no effect on the hardness. To treat permanent hard water, one must add carbonate ions directly, generally as sodium carbonate. Because many people use sodium carbonate for softening water for use with soap, sodium carbonate is often called `washing soda.'

The following activity is useful for students to experience treating hard water by addition of sodium carbonate, also called treatment by precipitation.

\subsubsection{Notes}
The addition of sodium carbonate will treat both permanent hard water and also temporary hard water.
